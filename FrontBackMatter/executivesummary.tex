%*******************************************************
% Executive Summary
%*******************************************************
\chapter*{Executive Summary}
\addcontentsline{toc}{chapter}{\numberline{}Executive Summary}

\par The use of fossil fuels is both harmful to the environment and unsustainable. It is linked to many problems worldwide such as greenhouse gas emissions, which cause climate change, and health issues such as cancer and respiratory problems. Furthermore, the earth's fossil fuel resources may be depleted by the end of the century.
\par In a legislative effort to reduce the use of fossil fuels in the Commonwealth of Massachusetts, the Global Warming Solutions Act (GWSA) was enacted in August of 2008. The GWSA created a framework for reducing heat-trapping emissions to levels that scientists believe gives humanity a chance of avoiding the worst effects of global warming. Pursuant to the GWSA, all sectors of the economy must reach a target of a 25\% reduction of Greenhouse Gas (GHG) emissions by 2020 and an 80\% reduction by 2050. As a result of the GWSA, many programs exist regarding energy efficiency and the use of renewable resources.
\par For our project we worked with the Department of Energy Resources (DOER) and their SAPHIRE and Renewable Thermal Programs that focus on reducing the fossil fuel dependence and greenhouse gas emissions in buildings. These programs had made an effort toward reducing fossil fuel use in the Commonwealth, but were lacking an evaluation and explanation of existing project processes. Evaluations and explanations of renewable thermal project processes would allow for future improvement while also providing information necessary to promote the DOER’s programs. 

\section*{Goal, Objectives and Methods}
\par The goal of our project was \goal. In order to achieve this goal, we completed the following three research objectives to produce case studies to evaluate pilot renewable thermal programs:
\par Our first objective was to \emph{identify the questions and concerns past project site leaders have had when considering implementing a renewable thermal project.} We utilized metrics formed from an interview with our sponsor to produce interview questions that we used when interviewing stakeholders at each project site. In order to identify questions and concerns past project site leaders have had we interviewed the on site project leaders and stakeholders including but not limited to: school superintendents, housing development directors, department of housing representatives, school facilities managers, and grant coordinators at the three case study sites. 
\par Our second objective was to \emph{determine project implementation processes and challenges of renewable thermal projects.} The application of renewable resources in buildings such as schools and public housing initiatives is relatively new in Massachusetts. Consequently, we found it important to produce an explanation of the entirety of the project process including decisions and how they were made, incentives and assistance that were provided, and any barriers encountered and how they were overcome. We interviewed those who were/are in charge of each site in regards to the implementation process of the DOER renewable thermal project at each project site.
\par Our third objective was to \emph{determine how these renewable thermal projects are performing in regards to cost and energy usage.} When producing these case studies, we desired quantitative data to illustrate the effectiveness of the project. In an effort to achieve this goal, we attempted to gather information from utility bills and projected values from feasibility studies as well as utilized existing energy auditing tools to collect and categorize numerical data from each project site.
\par Using the information obtained through the previous tasks, we produced case studies on the three renewable thermal projects to promote future renewable thermal energy projects. We then compiled this information for each of our three sites using the data we collected and organized from the interviews and qualitative data analysis we conducted.

\noindent
The three projects we focused on when completing these objectives and producing our deliverables were:
\begin{itemize}
  \item{\emph{Amherst College Bunker Building:} Installed biomass pellet boilers. Funded partially by the Mass CEC Pilot. Operational as of April 2015.}
  \item{\emph{Southern Berkshire Regional School:} Installed biomass pellet boilers. Funded partially by SAPHIRE and the MSBA. Nearing completion as of October 2015.}
  \item{\emph{Sudbury Public Housing Development:} Installed Air source heat pumps. Funded entirely by the SAPHIRE program. Operational as of January 2015.}
\end{itemize}

\section*{Results}
\par The findings from our fieldwork and data analysis are presented in this chapter. We first discuss the metrics used when collecting data, and then present the most relevant aspects of each of the following DOER-sponsored renewable thermal project sites we investigated: the Amherst College bunker building, the Southern Berkshire Regional School, and the Sudbury Public Housing Development. We then provide the list of findings we compiled from our data collection at each of the three project sites.
  \subsection*{Determination of Metrics}
  \begin{itemize}
    \item{Cost is a Metric for Assessing Renewable Thermal Projects}
    \item{Community Acceptance is a Metric for Assessing Renewable Thermal Projects}
    \item{Operational Logistics and Aesthetics are Metrics for Assessing Renewable Thermal Projects}
  \end{itemize}
  \subsection*{Site Findings}
  \begin{itemize}
    \item{Commissioning a feasibility study prior to beginning a renewable thermal project can be very valuable}
    \item{Failure to consider the context in which a renewable energy technology is being implemented can lead to poor performance and increased required maintenance}
    \item{Educating the community about new technology before installation can lead to community support}
    \item{Working with experienced engineers can make a big difference in the project timeline and post project effectiveness}
    \item{Improving a building’s energy efficiency improvements prior to or in addition to upgrading the heating system may lead to increased cost savings}
    \item{Public housing rules and regulations can be barriers when implementing a renewable technology project}
    \item{Failure to consider context when choosing a metering system can lead to problems with gathering data}
  \end{itemize}
  \par While many of our findings are specific to the project sites we studied, it is important to note that these findings can be applied to many renewable thermal project sites to come. From our findings we compiled a list of recommendations (located in the following chapter) that we presented to the DOER. It is our hope that the DOER will be able to use the recommendations to make improvements to future renewable thermal projects.

\section*{Recommendations}
Each recommendation is supported by our findings and review of relevant literature. It is important to note that our data analysis, and thus our recommendations are not without limitations. We were unable to provide an analysis of projected outcomes and actual outcomes relating to greenhouse gas emissions, renewable system energy efficiency, annual energy use, and annual fuel cost comparison of renewable systems and fossil fuel systems due to the lack of access to quantitative data at each site. Furthermore, our recommendations stem from site visits at only three project sites in a young program. For this reason, our recommendations reflect our analysis of qualitative data we collected including interviews and onsite visits from each of the three project sites. The recommendations are organized into major categories in the following sections.\\

\noindent
\textbf{Recommendations relating to informational material for potential project sites}
\begin{itemize}
  \item{We recommend that the DOER continue to create case studies to provide up to date information to potential project sites about existing renewable thermal projects}
  \item{We recommend that the DOER bring case studies when consulting with new project sites}
  \item{We recommend that the DOER provide examples of cost comparisons of fossil fuels and renewable energy including implementation and operational costs when consulting potential project sites}
\end{itemize}

\noindent
\textbf{Recommendation Relating to Community Education}
\begin{itemize}
  \item{We recommend that the DOER provide methods or examples for conducting outreach and education of the community on renewable thermal technologies when consulting potential project sites}
\end{itemize}

\noindent
\textbf{Recommendations Relating to Feasibility Studies}
\begin{itemize}
  \item{We recommend that a feasibility study be completed at each potential project site}
  \item{We recommend that potential project sites commission a feasibility study prior to beginning a renewable thermal project}
  \item{We recommend that the DOER strongly suggest or make it a requirement of their grant program that renewable thermal project sites commission a feasibility study prior to beginning a renewable thermal project}
  \item{We recommend that the DOER provide funding to commission a feasibility study that include a comparison of multiple renewable thermal heating and cooling systems}
  \item{We recommend that the DOER provide funding to commission a feasibility study that compares different types of metering systems and their application in the project site before providing funding to sites for metering systems}
\end{itemize}

\noindent
\textbf{Recommendations Relating to Experienced Engineers and Contractors}
\begin{itemize}
  \item{We recommend that project sites work with engineering firms and contractors that are familiar with renewable technologies}
  \item{We recommend that the DOER provide a list of engineering firms and contractors that are familiar with each type of renewable thermal technology to each project site}
  \item{We recommend that there be investment in educating contractors and electricians in renewable thermal technology}
\end{itemize}

\par Through the collection and analysis of data from project site leaders at three renewable thermal projects sponsored by the DOER, we achieved our goal, \goal, by producing three informational case studies. It is our hope that the case studies we produced will both promote the DOER renewable thermal programs and provide information to potential project site leaders that will, along with the list of recommendations, aid in improving renewable thermal project processes. As the number of successful renewable thermal projects increases in the Commonwealth of Massachusetts, greenhouse gas emissions and fossil fuel dependence will decrease, resulting in a healthier and more sustainable future.